\documentclass[12pt]{article}
\usepackage{Homework}

% Creates the header and footer.
\pagestyle{fancy}
\fancyhead[l]{Michael Hoon, $1006617$}
\fancyhead[c]{40.017 Probability \& Statistics HW5}
\fancyhead[r]{\today}
\fancyfoot[c]{\thepage}
\renewcommand{\headrulewidth}{0.2pt} %Creates a horizontal line underneath the header
\renewcommand{\footrulewidth}{0.2pt}
\setlength{\headheight}{15pt} %Sets enough space for the header
\newcommand{\HRule}[1]{\rule{\linewidth}{#1}}


\begin{document}

\title{ \normalsize \textsc{} 
        \\ [2.0cm]
		\HRule{1.5pt} \\
		\LARGE \textbf{\uppercase{40.017 Probability \& Statistics} 
        \HRule{2.0pt} \\ [0.6cm]
        \LARGE{Homework 5} \vspace*{10\baselineskip}}
		}
\date{\today}
\author{\textbf{Michael Hoon} \\ 1006617 \\ Section 2}

\maketitle 
\newpage


\section*{Question 1}

\subsection*{(a)}
Since we have 4 brands of spark plugs, with 5 plugs of each brands being tested, we have $N = 4 \times 5 = 20$, and $k = 4$ for each brand. The degrees of freedom are given by: $N - k = 20 - 4 = 16$, $k - 1 = 3$, and $N - 1 = 19$. Figure \ref{fig:1-excel} below shows the Excel screenshot of the missing entries. 

\begin{figure}[H]
    \centering
    \includegraphics[width=\textwidth]{Images/Q1Anova.png}
    \caption{Excel missing entries}
    \label{fig:1-excel}
\end{figure} 

\noindent The MSE is calculated using the relation $\text{MSE} = \frac{\text{SSE}}{N-k}$, and the SSA is calculated with the ANOVA Identity $\text{SST} = \text{SSA} + \text{SSE}$. Lastly, MSA is obtained with the relation $\text{MSA} = \frac{\text{SSA}}{k-1}$. The answers are given to 3 decimal places. 

\subsection*{(b)}

Let $\mu_i$ denote the mean performance of the $i$th brand of spark plugs, $\forall i \in {1, 2, 3, 4}$. We define the hypothesis: \begin{align*}
    H_0 &: \mu_1 = \mu_2 = \mu_3 = \mu_4 \\ 
    H_1 &: \text{at least 1 pair of the } \mu_{i} \text{'s are different}
\end{align*}

\subsection*{(c)}

From Figure \ref{fig:1-excel}, since the p-value $= 0.206927 > \alpha = 0.05$, we do not have sufficient evidence to reject $H_0$ at the 95\% significance level, and hence we conclude that the mean performance of the 4 brands of spark plugs are equal. 

\newpage

\section*{Question 2}

\subsection*{(a)}

Using the Bonferroni Method, we have $m = \displaystyle \binom{k}{2}$ pairs involved in testing. Since we have 4 sites in total, then we have $m = \displaystyle\binom{4}{2} = 6$ pairs. 

\subsection*{(b)}

\begin{figure}[H]
    \centering
    \includegraphics[width=\textwidth]{Images/Q2water.png}
    \caption{Excel values}
    \label{fig:2-excel}
\end{figure} 

\noindent The formula for SSA is given by \begin{align} \nonumber
    \text{SSA} &= \sum_{i} (\bar{y}_i - \bar{\bar{y}})^{2} \\ \label{eq:2-ssa}
    &= \sum_{i=1}^{4} n_i (\bar{y}_i - \bar{\bar{y}})^{2}
\end{align} From the table, we find that $\text{SSA} = 40.46382259$. From Equation \ref{eq:2-ssa}, we use the calculated $\bar{y}_i$'s as well as the grand mean to form a quadratic in $x$. We have the relation: \begin{align*}
    40.46382259 = 12 \left( \frac{408.91 + x }{12} - \bar{\bar{y}} \right)^{2} + 8\left( 40.10375 - \bar{\bar{y}} \right)^{2} + 10\left( 38.892 - \bar{\bar{y}} \right)^{2} + 9\left( 39.065556 - \bar{\bar{y}} \right)^{2}
\end{align*} where $\bar{\bar{y}} = (1470.25 + x) / 39$. Restructuring the equation, we have \begin{align} \nonumber
    12 \left( \frac{408.91 + x }{12} - \bar{\bar{y}} \right)^{2} + 8\left( 40.10375 - \bar{\bar{y}} \right)^{2} + 10\left( 38.892 - \bar{\bar{y}} \right)^{2}  \\ \label{eq:2-x}
    + 9\left( 39.065556 - \bar{\bar{y}} \right)^{2} - 40.46382259 = 0 
\end{align} We can now solve for $x$ in Equation \ref{eq:2-x} by using an online solver. First, plotting the equation in desmos we get the following graph: 

\begin{figure}[H]
    \centering
    \includegraphics[width=\textwidth]{Images/Q2Desmos.png}
    \caption{Desmos Graph}
    \label{fig:2-desmos}
\end{figure} From here, we can see that the two values of $x$ are 38.83 and 86.763. To determine which of the two is the correct solution, we plug these values into the Excel file and run the ANOVA test again: 

\begin{figure}[H]
    \centering
    \includegraphics[width=\textwidth]{Images/Q2firstx.png}
    \caption{ANOVA for $x = 38.83$}
    \label{fig:2-firstx}
\end{figure} 

\noindent We see that for $x = 38.83$, we obtain an ANOVA table with values which correspond to the original ANOVA table, and we can conclude that the correct solution for $x$ is indeed 38.83.

\begin{figure}[H]
    \centering
    \includegraphics[width=\textwidth]{Images/Q2secondx.png}
    \caption{ANOVA for $x = 86.763$}
    \label{fig:2-secondx}
\end{figure} 

\noindent On the other hand, the ANOVA table for $x = 86.763$ does not match the values given in the original ANOVA table, hence we \textbf{reject this solution}. 

\newpage

\section*{Question 3}

The 90\% Prediction Interval for the winning distance corresponding to the year 1940 is given by: \begin{align*}
    \left( \hat{y}^{*}_{1940} - t_{n-2, \; \alpha / 2} \cdot s \sqrt{1 + \frac{1}{n} + \frac{(x^{*}_{1940} - \bar{x})^{2}}{(n-1)s_x^{2}}}, \; \hat{y}^{*}_{1940} + t_{n-2, \; \alpha / 2} \cdot s \sqrt{1 + \frac{1}{n} + \frac{(x^{*}_{1940} - \bar{x})^{2}}{(n-1)s_x^{2}}}\right)
\end{align*} where $s$ represents the Mean Squared Error (MSE) and we have $n = 29$ observations, with $\alpha = 0.10$. The values for each variable used is calculated in Excel in Figure \ref{fig:3-excel}. 

\begin{figure}[H]
    \centering
    \includegraphics[width=\textwidth]{Images/Q3Regression.png}
    \caption{Excel Calculations for variables}
    \label{fig:3-excel}
\end{figure}

\noindent With this, we obtain the 90\% Confidence Interval in 3 decimal places: \begin{align*}
    \left( 15.120, 16.462 \right)
\end{align*}

\newpage

\section*{Question 4}

\subsection*{(a)}

The formula for SSE for Regression is given by: \begin{align*}
    \text{SSE} = \sum_{i=1}^{n} (y_i - \hat{y}_i)^{2} 
\end{align*} Let $n$ be the number of rows of data, and $k$ be the total number of predictors. AIC is defined as follows: \begin{align*}
    \text{AIC} = n \ln \left( \frac{\text{SSE}}{n} \right) + 2 (m+1)
\end{align*} with $m$ representing the subset of number of predictors in the model. The Excel calculated values are given in Figure below. 

\begin{figure}[H]
    \centering
    \includegraphics[width=\textwidth]{Images/Q4SSE.png}
    \caption{SSE and AIC Values}
    \label{fig:4-sseaic}
\end{figure} 

\noindent The SSE and AIC values for each model is given in Table below (3 decimal places). 

\begin{table}[H]
    \centering
    \begin{tabular}{| c | c | c | c |} \hline
        \textbf{Model} & $Y$ vs $x_1$ & $Y$ vs $x_2$ & $Y$ vs $x_1, \; x_2$ \\ \hline 
        \textbf{SSE} & 32.620 & 52.343 & 32.570 \\ \hline 
        \textbf{AIC} & 8.276 & 21.517 & 10.233 \\ \hline
    \end{tabular}
    \caption{Table of SSE and AIC Values}
    \label{5-sseaic}
\end{table}

\noindent The AIC values are computed as follows (3 decimal places): \begin{align*}
    \text{AIC}_{x_1} &= 28 \ln \left( \frac{32.61954}{28} \right) + 2 (1+1) \approx 8.276 \\
    \text{AIC}_{x_2} &= 28 \ln \left( \frac{52.34314}{28} \right) + 2 (1+1) \approx 21.517\\ 
    \text{AIC}_{x_1, \; x_2} &= 28 \ln \left( \frac{32.57028307}{28} \right) + 2 (2+1) \approx 10.233
\end{align*} 

\subsection*{(b)}

The best model corresponds to the one with the lowest AIC, which is given by \begin{align*}
    \min \left\{ 8.276, 21.517, 10.233 \right\} = 8.276
\end{align*} This value corresponds to model (1), which is $Y \text{ vs } x_1$. 

\newpage

\section*{Question 5}

The Python code is modified to conduct the bootstrap resampling $10^{4}$ times, and the distribution is obtained from the means of each resampling process. Then, the values are sorted in ascending order into a Python list (using list comprehension), and a Histogram of the distribution is plotted, with the corresponding 'L' and 'U' markers to denote the Lower and Upper bounds of the bootstrap Confidence Interval for the true mean. \\ 

\noindent Since we are finding the 99\% Confidence Interval, only the lowest $ \frac{0.01 \times 10^{4}}{2} = 50 $ and highest 50 values were excluded from the interval. Figure \ref{fig:5-bootstrap} below shows the Python Code used.

\begin{figure}[H]
    \centering
    \includegraphics[width=\textwidth]{Images/Q5Bootstrap.png}
    \caption{Python Code for Bootstrap}
    \label{fig:5-bootstrap}
\end{figure} 

\noindent From the Python code, we obtain a 99\% Confidence Interval (3 decimal places) of \begin{align*}
    \left[ 22.182, 28.879 \right]
\end{align*}

\end{document}
