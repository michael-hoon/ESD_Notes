
\documentclass[12pt]{article}

\usepackage{Homework}

% Creates the header and footer.
\pagestyle{fancy}
\fancyhead[l]{Michael Hoon, $1006617$}
\fancyhead[c]{40.017 Probability \& Statistics HW1}
\fancyhead[r]{\today}
\fancyfoot[c]{\thepage}
\renewcommand{\headrulewidth}{0.2pt} %Creates a horizontal line underneath the header
\renewcommand{\footrulewidth}{0.2pt}
\setlength{\headheight}{15pt} %Sets enough space for the header
\newcommand{\HRule}[1]{\rule{\linewidth}{#1}}


\begin{document}

% \title{Another fancyhdr demo}
% \author{\texttt{tex.stackexchange} \textit{et al}}
% \maketitle
% \newpage


\title{ \normalsize \textsc{} 
        \\ [2.0cm]
		\HRule{1.5pt} \\
		\LARGE \textbf{\uppercase{40.017 Probability \& Statistics} 
        \HRule{2.0pt} \\ [0.6cm]
        \LARGE{Homework 1} \vspace*{10\baselineskip}}
		}
\date{\today}
\author{\textbf{Michael Hoon} \\ 1006617}

\maketitle 
\newpage


\section*{Question 1}
\subsection*{(a)}
Let $A$ be the random variable denoting the number of questions chosen from section A. Since the student 'samples' the question without replacement, $A$ follows a hypergeometric distribution $A \sim \text{hypgeo}(12,9,9)$. The pmf $f(x)$ is given by:

\begin{equation*}
    f(x) = \mathbb{P}(X = x) = \frac{\binom{j}{x}{\binom{k}{n-x}}}{\binom{j+k}{n}}
\end{equation*}

\noindent for 6 questions from section A, we have $\mathbb{P}(X=6)$:

\begin{align*}
    \mathbb{P}(X=6) &= \frac{\binom{9}{6}\binom{9}{6}}{\binom{18}{12}} \\ 
    &= 0.380090 \\
    &\boxed{\approx 0.380}
\end{align*}

\subsection*{(b)}
We need to consider two cases, one where 5 questions come from Section A and the other 7 from Section B, and vice versa. i.e. 

\begin{align*}
    \mathbb{P}(A = 5) + \mathbb{P}(A = 7) &= \frac{\binom{9}{5}\binom{9}{7}}{\binom{18}{12}} + \frac{\binom{9}{7}\binom{9}{5}}{\binom{18}{12}} \\ 
    &= 0.4886877 \\ 
    &\boxed{\approx 0.489}
\end{align*}

\section*{Question 2}

\subsection*{(a)}

To find the probability that at least 1 man receives his own hat, we consider the complementary case of no man receiving his own hat, given by $D_6$. Thus:

\begin{align*}
    \mathbb{P}(\text{'at least 1 man receives his own hat'}) &= 1 - \frac{D_{6}}{6!} \\ 
    &= \\
    & \boxed{\approx }
\end{align*}

\subsection*{(b)}

Again we consider the complementary case that no man receives his own hat, or only 1 man receives his own hat. Thus:

\begin{align*}
    \mathbb{P}(\text{'at least 2 men receives their own hats'}) &= 1 - \underbrace{\frac{D_{6}}{6!}}_{\text{0 man receives his own hat}} - \underbrace{\frac{6 \times D_{5}}{6!}}_{\text{1 man receives his own hat}} \\ 
    &= \\
    & \boxed{\approx }
\end{align*}

\subsection{(c)}

The probability that at least 5 men receive their own hats can be interpreted as every man receiving their own hats, since if 5 man receive their own hats, the last man will automatically also receive his own hat. In this case, the probability is thus 1:


\begin{align*}
    \mathbb{P}(\text{'at least 5 men receives their own hats'}) = \boxed{1}
\end{align*}



\section*{Question 3}

\section*{Question 4}

\section*{Question 5}

Let $X \sim \text{negbin}(n,p)$. The MGF of $X$ is:

\begin{equation*}
    M_X (t) = 
\end{equation*}

\section*{Question 6}

\section*{Question 7}

\end{document}
